\documentclass[french]{article}

\usepackage[french]{babel}

\title{TP 1 : mesure de puissance sur un ordinateur}
\author{Groupe LTB}
\date{27 septembre 2022}

\begin{document}

\maketitle

\section{Objectifs}

On cherche à mesurer la consommation énergétique d'un ordinateur dans différentes configurations de travail, afin de pouvoir le modéliser par un circuit électrique plus simple.

L'idéal serait de pourvoir modéliser l'ordinateur comme un circuit d'impé\-dance complexe connue et consitué de dipôles simples comme des bobines, des résistors ou des condensateurs.

\section{Matériel}

On dispose d'un wattmètre (référence IDKMPM70), d'un GBF, d'un oscilloscope et d'un multimètre.
On notera les mesures sur papier, et ces mesures seront ultérieurement analysées par Malo. Le code \texttt{python} de présentation des données sera disponible sur le dépôt dès que possible.

\section{Manipulation}

En utilisant le logiciel y-cruncher, on fait fonctionner l'ordinateur avec des calculs de différentes difficultés. On mesure en fonction du temps (en prenant en compte le temps de réponse du wattmètre) la consommation en puissance de l'ordinateur.

\section{Résultats}

\textit{Cette section est en cours d'écriture...}

\end{document}
