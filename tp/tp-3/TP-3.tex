\documentclass[french]{article}

\usepackage[utf8]{inputenc}
\usepackage[T1]{fontenc}
\usepackage[frenchb]{babel}
\usepackage{amssymb}
\usepackage{esvect}
\usepackage{physics}
\usepackage{siunitx}
\sisetup{
    detect-all,
    output-decimal-marker={,},
    group-minimum-digits = 3,
    group-separator={~},
    number-unit-separator={~},
    inter-unit-product={~}
}
\newcommand{\cel}{\degreeCelsius}
% \newcommand{\grad}{\vv{\mathrm{gra}}\mathrm{d}}
\newcommand{\bs}{\xspace$\backslash$\xspace}
\newcommand{\moy}{\expval}

\title{TP 3 : grandeurs thermodynamiques\\
d'un micro-ordinateur}
\author{Groupe 10, MP*}

\begin{document}

\maketitle

\section{Objectifs}

On cherche à déterminer la valeur des grandeurs physiques suivantes :
\begin{itemize}
    \item La masse équivalente en eau $m_{\text{calo}}$ d'un calorimètre
    \item La masse $m$, les dimensions spatiales et la capacité thermique $C$ d'un micro-ordinateur Raspberry Pi 3
\end{itemize}
\section{Matériel}

\begin{itemize}
    \item Raspberry Pi
    \item Alimentation continue
    \item Carte SYSAM
    \item Calorimètre
\end{itemize}

\section{Manipulations}

\subsection{Séance 1}

\subsubsection{Masse en eau du calorimètre}

On suit le protocole suivant pour déterminer $m_\text{calo}$.
\begin{enumerate}
    \item Le laboratoire est initialement à la température $T_\text{a} = \SI{20}{\cel}$
    \item On met dans le calorimètre $m_\text{eau} = \SI{200}{g}$ d'eau chaude à $T_\text{c} = \SI{52}{\cel}$
    \item On mesure après thermalisation (pas trop longue pour éviter les fuites thermiques) la température de l'eau $T_\text{f} = \SI{47}{\cel}$
    \item La masse équivalente en eau du calorimètre est donnée par
    $$m_\text{calo} = \frac{T_\text{c} - T_\text{f}}{T_\text{f} - T_\text{a}} m_\text{eau} = \SI{37}{g}$$

\end{enumerate}

En prenant $c_{\text{eau}} = \SI{4,18}{kJ.K^{-1}.kg^{-1}}$ on en déduit $C_{\text{calo}} = \SI{167}{J/K}$.

\subsubsection{Capacité thermique de l'ordinateur}

On mesure les grandeurs $I_{\text{inst}}$, $I_{\text{eff}}$ et $T$ en fonction du temps avec 1000 points uniformément répartis sur une durée d'acquisition de 10 min (une mesure toute les 600 ms).
\begin{enumerate}
    \item Avant calculs, $\moy{T}_\text{i} = \SI{18,98}{\cel}$ et la valeur moyenne de $I_{\text{inst}}$ est de 288 mA
    \item Lancement des calculs à $t = \SI{1}{min}$
    \item Fin des calculs à $t = \SI{381,6}{s}$. Pendant les calculs, on a
    \begin{itemize}
        \item la tension $U = \SI{5}{V}$
        \item l'intensité $\moy{I_{\text{inst}}} = \SI{381}{mA}$
        \item la puissance $\moy{P} = \SI{1,90}{W}$
        \item le travail électrique $W = \SI{612}{J}$
    \end{itemize}
    \item Régime stationnaire de thermalisation : sur les 100 dernières secondes, $\moy{T}_\text{f} = \SI{20,63}{\cel}$
\end{enumerate}

La valeur de $C$ est obtenue par

$$C = \frac{W}{\moy{T}_\text{f} - \moy{T}_\text{i}} - C_{\text{calo}} = \SI{204}{J/K}$$

\subsubsection{Masse de l'ordinateur}

La grandeur donnée par le vendeur est $m = \SI{45}{g}$.

Avec la valeur obtenue pour $C$, on en déduit la capacité thermique massique moyenne de l'ordinateur

$$c = \frac{C}{m} = \SI{4,5}{kJ.K^{-1}.kg^{-1}}$$

\underline{\textbf{Commentaires :}}
\begin{itemize}
    \item La valeur tabulée de la capacité thermique massique du silicium, composant principal des circuits imprimés est $c_\text{silicium} = \SI{0,7}{kJ.K^{-1}.kg^{-1}}$
    \item La valeur tabulée de la capacité thermique massique du PVC (souple), qui compose la gaine des câbles connectés à l'ordinateur est $c_\text{PVC} = \SI{0,9}{kJ.K^{-1}.kg^{-1}}$
    \item La masse donnée par le vendeur ne prend pas en compte l'enceinte en plastique (PCV rigide), ni les câbles (PVC souple) connectés à l'ordinateur
\end{itemize}

\end{document}
