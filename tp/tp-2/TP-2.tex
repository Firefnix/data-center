\documentclass[french]{article}

\usepackage[utf8]{inputenc}
\usepackage[T1]{fontenc}
\usepackage[frenchb]{babel}
\usepackage{amssymb}
\usepackage{esvect}
\usepackage{siunitx}
\sisetup{
    detect-all,
    output-decimal-marker={,},
    group-minimum-digits = 3,
    group-separator={~},
    number-unit-separator={~},
    inter-unit-product={~}
}
\newcommand{\cel}{\degreeCelsius}
\newcommand{\grad}{\vv{\mathrm{gra}}\mathrm{d}}
\newcommand{\bs}{\xspace$\backslash$\xspace}

\title{TP 2 : influence de la température\\
sur un micro-ordinateur}
\author{Groupe LTB}

\begin{document}

\maketitle

\section{Objectifs}

On cherche à étudier l'influence de la quantité de calculs et de la température sur la consommation électrique d'un micro-ordinateur procédant à divers calculs. L'objectif est de déterminer la fonction $f$ de la température extérieure $T_{ext}$ et de la quantité de calculs $K$ donnant $I = f(T_{ext}, K)$.

\section{Matériel}

\begin{itemize}
    \item Raspberry Pi
    \item Alimentation continue
    \item Carte SYSAM
    \item Moyens d'influencer la température : étuve, sèche-cheveux, bac à glace...
\end{itemize}

\section{Manipulations}

\subsection{Séance 1}

Après une prise en main du Raspberry Pi et la réalisation du montage de mesure de sa consommation, on a réalisé mesures en intensité efficace (la tension est constante à $U = \SI{5,00}{V}$). Les valeurs vont de $I_{min} = \SI{200}{mA}$ au repos (sans calculs autre que ceux réalisés par le système d'exploitation), à $I_{max} = \SI{600}{mA}$ aux alentours de 4000 kcps (trois fenêtres à 1300 kcps).

\subsection{Séance 2}

Le chauffage du micro-ordinateur à l'aide d'un sèche-cheveux entraîne, pour la même quantité de calculs (1000 kcps) une augmentation l'intensité efficace, donc de la puissance consommée : de $\SI{392}{mA}$ à température ambiante, le courant passe à $\SI{458}{mA}$ après un chauffage d'une trentaine de secondes, pour redescendre à $\SI{424}{mA}$ après deux minutes de refroidissement à l'air libre.

\subsection{Séance 3}

Il s'agit essentiellement de la même manipulation que la séance 2, mais avec un appareil de chauffage plus perfectionné (étuve). On cherche à obtenir un résultat plus quantitatif que celui de la séance 2, qui n'était que qualitatif. Les mesures donnent au repos, pour 1000 kcps, les résultats suivants : on donne ici l'intensité moyenne consommée $I$, en mA, en fonction de T (en °C) et de K (en kcps).

\begin{table}[ht]
    \centering
    \begin{tabular}{|c|c|c|c|c|c|} \hline
    \textbf{K \bs T } &   31  &   35  &   41  &   46  &   56  \\ \hline
            0         &$ 268 $&$ 268 $&$ 267 $&$ 274 $&$ ... $\\ \hline
            1000      &$ 304 $&$ 310 $&$ ... $&$ 310 $&$ 310 $\\ \hline
    \end{tabular}
\end{table}


\end{document}
