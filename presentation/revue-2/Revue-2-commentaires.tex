\documentclass{article}

\usepackage[french]{babel}
\usepackage[utf8]{inputenc}
\usepackage[T1]{fontenc}

\usepackage[locale = FR]{siunitx} % les unités PROPRES
\sisetup{inter-unit-product = \ensuremath{{}\cdot{}}}

\newcommand*{\eff}{_\text{eff}}
\newcommand{\fleche}{$\Rightarrow {}$}

\title{Commentaires sur la revue 2}
\author{Groupe 10, MP*}

\begin{document}

\maketitle

\textit{Ce document est une liste des remarques et suggestions qui ont été faites par M. Delannoy lors de la 2 revue de TIPE.}

\vspace{1cm}

\section*{À traiter}

\begin{itemize}
    % \item Expliquer d'où sortent les valeurs du modèle par circuits \fleche une formule pour un type de circuit (e.g. le $L/R$)
    \item Diapositive 5 : suppr. « définition de la température difficile » et clarifier la raison d'abandon du modèle par circuit (on n'a pas accès à R(K, T) et à C(K, T))
    \item Mettre les courbes $T(t)$ et $I(t)$
\end{itemize}

\section*{Résolus pour la revue suivante}

\begin{itemize}
    \item Présenter d'abord des « coupes » plutôt que de montrer le graphe en volume directement \fleche $P(K)$, $P(T)$
    \item Parler de masse volumique \textbf{moyenne}
    \item Parler de conductivité thermique \textbf{moyenne}
    \item Parler de capacité thermique massique \textbf{moyenne}
    \item Faire apparaître l'objectif du TIPE
    \item « Vernier » : on s'en fiche
    \item Thermomètre : dire que le capteur est dans le boîtier du Raspberry Pi
    \item Deux C.S. seulement : $\SI{1,7e2}{J/K}$
    \item Dire « inconvénients » plutôt que « désavantages »
    \item « Modèle »: pas le bon terme pour le Raspberry Pi \fleche les circuits étaient une impasse
    \item Mettre la masse totale avec carcasse
    \item Ne pas désigner une équation aux dérivées partielles comme étant une équation différentielle
    \item Protocole pour la masse en eau : préciser qu'on est sans le Raspberry Pi
    \item Parler de \textbf{diffusion} thermique plutôt que de conduction
    \item On \textbf{modifie} la température
    \item On mesure l'intensité \textbf{moyenne}
    \item Pour la vue avec le plan obtenu par régression linéaire : orienter la vue de manière à ne voir que la tranche du plan
\end{itemize}

\section*{Commentaires complémentaires}

\textit{Ces commentaires ont étés faits à Clément et Julien pendant leurs deuxièmes revues respectives. On n'a gardé ici que les commentaires pertinents dans le cadre d'un TIPE de physique ou d'informatique.}

\vspace{1em}

\begin{itemize}
    \item Ne pas aller trop vite dans la présentation (surtout au début quand on introduit le sujet)
    \item Ne pas dire qu'un élément, un contept ou une démonstration est « facile » ou « évidente » : cela n'enrichit pas la présentation si l'examinateur comprend, et est vexant s'il ne comprend pas
    \item Ne pas hésiter à illustrer en donnant des exemples
    \item Ce qui est proche du cours est à enlever, ou à ne mentionner que succintement
    \item Dans le cadre d'un TIPE souhatant améliorer un système ou un calcul par rapport à une solution naïve, il est nécessaire de comparer la solution à laquelle on a abouti avec la solution naïve
\end{itemize}

\end{document}
