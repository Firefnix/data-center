\documentclass{article}

\usepackage[french]{babel}
\usepackage[utf8]{inputenc}
\usepackage[T1]{fontenc}

\usepackage[locale = FR]{siunitx} % les unités PROPRES
\sisetup{inter-unit-product = \ensuremath{{}\cdot{}}}

\newcommand*{\eff}{_\text{eff}}
\newcommand{\fleche}{$\Rightarrow {}$}

\title{Commentaires sur la revue 3}
\author{Groupe 10, MP*}

\begin{document}

\maketitle

\textit{Ce document est une liste des remarques et suggestions qui ont été faites par M. Delannoy lors de la 3 revue de TIPE (Malo) réalisée le 30 mai.}

\vspace{1cm}

\begin{itemize}
    \item Mettre le nom
    \item Soit dessiner les circuits, soit ne pas en parler
    \item Fusionner des diapos (e.g 5 et 6, etc.)
    \item Introduire les notations (e.g. $\rho$)
    \item Dire « carte d'aquisition » plutôt que « carte Sysam »
    \item Être clair sur le fait que l'ordinateur soit éteint ou pas
    \item Chiffres significatifs (températures)
    \item Ne pas laisser deux caractères sur une seule ligne
    \item Tension $\SI{5,0}{V}$, pas seulement $\SI{5}{V}$
    \item Pertes thermiques : possible prise en compte avec une loi de Newton (température exponentielle, mesure de la constante de temps)
    \item Masse en eau : à inclure (succintement) dans le protocole
    \item \textbf{Texte plus gros sur les graphes Python}
    \item Ajouter un \texttt{plt.grid()}
    \item Formule toute faite : $\sigma_\text{glissant} = \frac{\sigma}{\sqrt{n}}$
    \item Courbe orange à mettre en rouge
    \item « Micro-plan » pas forcément utile
    \item Parler de constante additive (pour $K$)
    \item Clarifier le « au repos » : « on note $K$ ... » puis « au repos, on prend $K = 0$ »
    \item Pas de reliage de points : Gros points expérimentaux et régression linéaire $P = b \times K + c(T)$
    \item Le 4 \% est un rapport d'ordonnées à l'origine
    \item Mettre les formules de régression linéaires à côté des graphes
    \item Il y a bien un écart relatif par point, et ensuite on fait la moyenne
    \item $D$ est dans la dérivéé (enfin $\lambda$ l'est) : la dépendance en $T$ n'est pas forcément une bonne idée
    \item Faire des simulations sur plus d'une heure (e.g. une journée voire une semaine : il faut un effet notable)
    \item Montrer la répartition spatiale de la situation avant toute chose
    \item Avoir un temps en heures plutôt qu'en milliers de secondes
    \item Dire « on fait varier les positions et la quantité de calculs »
    \item Terminer proprement (pas abruptement) et expliquer, commenter les résultats obtenus
\end{itemize}

\end{document}
