\documentclass{article}

\usepackage[french]{babel}
\usepackage[utf8]{inputenc}
\usepackage[T1]{fontenc}

\usepackage[locale = FR]{siunitx} % les unités PROPRES
\sisetup{inter-unit-product = \ensuremath{{}\cdot{}}}

\newcommand*{\eff}{_\text{eff}}
\newcommand{\fleche}{$\Rightarrow {}$}

\title{Commentaires sur la revue 1}
\author{Groupe LTB}

\begin{document}

\maketitle

\textit{Ce document est une liste des remarques et suggestions qui ont été faites par M. Delannoy lors de la 1 revue de TIPE, qui s'est tenue le 22 nov. 2022.}

\vspace{1cm}

\begin{itemize}
    \item Carte des data-center (motivation)
    \item Photo wattmètre + explication du protocole (influence de K, justification de l'utilisation du Raspberry)
    \item Signe de $\varphi$ pour les modèles simples \fleche manipulation
    \item Écrire le protocole (ne pas seulement le dire)
    \item Définir proprement $K$
    \item Chiffres significatifs, ou barres d'incertitude (mieux)
    \item Virgule en séparateur des milliers
    \item Parler de $P\eff$ au lieu de $I\eff$
    \item Mesurer $P\eff(K)$ en faisant croître K progressivement puis pour les mêmes valeurs mais en décroissant (noter les temps de pause) \fleche hystérésis
    \item Demander au labo un thermomètre à four (avec câble)
    \item Thermomètres dans l'étuve \fleche près des parois et près du Raspberry
    \item Coeff. $h$ de la loi de Newton (?)
    \item T dépend de la puissance évacuée \fleche voir cours loi de Newton
    \item Pour les modèles à plusieurs ordinateurs, expliquer \fleche pavé supposé homogène, de capacité thermique $C$
    \item Le modèle 3D est \textit{entre} les Raspberry
    \item Faire deux modèles \fleche Python (fait main) et SolidWorks
    \item Faire un modèle à 1D (ordinateurs en ligne), puis
    \item Faire un modèle à 2D (rectangle par exemple), puis éventuellement
    \item Faire un modèle à 3D
    \item Dire que le problème est \textbf{bouclé}
    \item Se baser sur le DM n°5 pour la modélisation
\end{itemize}

\end{document}
