\documentclass[french]{article}

\usepackage[utf8]{inputenc}
\usepackage[T1]{fontenc}
\usepackage[frenchb]{babel}
\usepackage{xspace}

\title{Revue 1 : contenu}
\author{Groupe 10, MP*}

\begin{document}

\maketitle

\section{En-tête}

\begin{itemize}
    \item \textbf{Titre :} Consommation électrique d'un data-center
    \item \textbf{Ancrage au thème :}
    \item \textbf{Motivation :}
\end{itemize}

\section{Diapositives}

\begin{enumerate}
    \item Titre
    \item Annonce du plan
    \item Modélisation par des circuits éléctriques simples
    \item Influence de la quantité de calculs
    \item Influence de la température
\end{enumerate}

\section{Plan}

\begin{enumerate}
    \item Modélisation par des circuits éléctriques simples
    \item Influence de la quantité de calculs
    \item Influence de la température
\end{enumerate}

\section{À faire plus tard}

\begin{itemize}
    \item Modéliser par une simulation
    \begin{itemize}
        \item Déterminer l'équation de la chaleur à 3D
        \item Trouver une cpacité thermique $C$ équivalente et une conductivité thermique $\lambda$ équivalente
        \item Pour $C$, mettre le Raspberry dans un colorimètre avec une résistance

    \end{itemize}
\end{itemize}

\end{document}
