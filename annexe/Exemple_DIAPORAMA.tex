
	\documentclass[a4paper,11pt]{beamer}
%	\documentclass[a4paper,11pt,handout]{beamer}

%%%%%%%%%%%%%%%%%%%%%%%%%%%%%%%%%%%%%%%%%%%%%%%%%%%%%%
%%%%%      Les packages de base                %%%%%%%
%%%%%%%%%%%%%%%%%%%%%%%%%%%%%%%%%%%%%%%%%%%%%%%%%%%%%%

\usepackage[utf8]{inputenc}  % pour taper directement les accents
\usepackage[T1]{fontenc}     % pour inclure les polices avec accents et gérer la césure des mots
\usepackage[french]{babel}	 % pour utiliser les règles de la typographie française
\usepackage{mathtools,amssymb,amsfonts}	% plus de math :-)
\usepackage{graphicx}	% c'est pour \includegraphics et \graphicspath
\usepackage{lastpage}   % \cfoot{\thepage $/$ \pageref{LastPage}}
\usepackage{ifthen}     % pour \ifthenelse
\usepackage{eurosym}    % pour \euro

\usepackage[locale = FR]{siunitx} % les unités PROPRES
\sisetup{inter-unit-product = \ensuremath{{}\cdot{}}}

\usepackage{pgfplots}  % environnement axis
\pgfplotsset{compat=1.15}

\usepackage{tikz}
\usetikzlibrary{calc}  % pour faire des calculs sur les coordonnées ($(A)+(45:3)$)

\usepackage{tikzelec}  % package maison pour dessiner les circuits électroniques

%%%%%%%%%%%%%%%%%%%%%%%%%%%%%%%%%%%%%%%%%%%%%%%%%%%%%%
%%%%%  Pour numeroter les pages                %%%%%%%
%%%%%%%%%%%%%%%%%%%%%%%%%%%%%%%%%%%%%%%%%%%%%%%%%%%%%%

\setbeamertemplate{navigation symbols}{}
%=======================================================
% Pour numéroter les diapos :
\setbeamertemplate{footline}[frame number]
%=======================================================
% Pour numéroter les diapos en choissant le numéro de la dernière
% dans ce cas enlever la commande \setbeamertemplate{footline}[page number]
% et utiliser les deux lignes suivantes
\newcommand {\NumeroDerniereDiapo} {22}
%\addtobeamertemplate{footline}{\hfill\insertframenumber\,/\,\NumeroDerniereDiapo\hspace{2mm}\null\vspace{1mm}}
%=======================================================

%%%%%%%%%%%%%%%%%%%%%%%%%%%%%%%%%%%%%%%%%%%%%%%%%%%%%%
%%%%%  Pour positionner les choses où on veut  %%%%%%%
%%%%%%%%%%%%%%%%%%%%%%%%%%%%%%%%%%%%%%%%%%%%%%%%%%%%%%

\usepackage[absolute,showboxes,overlay]{textpos}
\textblockorigin{10mm}{10mm} % origine des positions
\setlength{\TPHorizModule}{1mm} % échelle horizontale
\setlength{\TPVertModule}{\TPHorizModule} % échelle verticale identique à l'horizontale

% à ajouter dans les frames pour positioner les objets
% \begin{textblock}{largeur}(x,y) (les nombres sont en mm) :
%
%%\TPshowboxestrue  % (par défaut) les boites sont visibles
%%\TPshowboxesfalse % décommenter pour faire disparaitre les boites
%
%\begin{textblock}{115}(-3,10)
%bla bla
%\end{textblock}
%


\newenvironment{tikzgrille}[1][0]{
% 1 pour afficher la grille / 0 pour ne pas l'afficher = option par défaut
\begin{textblock}{180}[.5,.5](54,38)
\begin{tikzpicture}
\draw (-9,-9) rectangle (9,9);
\ifthenelse {#1=1} {\begin{scope}[magenta!60]
\draw (-9,-9) grid (9,9);
\fill (0,0) circle (.05);
\fill (1,0) circle (.05);
\fill (0,1) circle (.05);
\node[anchor=text] at (0.1,0.2) {\footnotesize $(0,0)$};
\node[anchor=text] at (1.1,0.2) {\footnotesize $(1,0)$};
\node[anchor=text] at (0.1,1.2) {\footnotesize $(0,1)$};
\end{scope}} {} }
{\end{tikzpicture}\end{textblock}}

%%%%%%%%%%%%%%%%%%%%%%%%%%%%%%%%%%%%%%%%%%%%%%%%%%%%%%
%%%%%%%%%%%%%%%%%%%%%%%%%%%%%%%%%%%%%%%%%%%%%%%%%%%%%%

\begin{document}

%=====================================================
%=====================================================
%=====================================================

% ATTENTION : l'option [fragile] permet d'utiliser \verb mais neutralise \uncover
\begin{frame}[fragile]

\frametitle{Quelques conseils}
\framesubtitle{Pour un TIPE heureux\,\dots}

\begin{itemize}\setlength{\itemsep}{5mm}

\item Numérotez vos slides !\hspace{5mm}{\tiny Il est possible de fixer le numéro de la dernière page  pour ajouter des annexes après la \frquote{dernière page} (voir commandes dans le préambule de ce fichier).}

\item En physique, mettez une photo où on vous voit (seul)

avec le dispositif expérimental (avec des explications).

\item En physique, pour les unités, utiliser le package \texttt{siunitx} :
% ATTENTION pour utiliser \verb il faut mettre [fragile] en option après \begin{frame}

\medskip

\verb|\SI{1.38e-23}{J.K^{-1}}| \ donne \ \SI{1.38e-23}{J.K^{-1}}

\smallskip

pour les valeurs sans unité : \verb| \num{e-4}  \num{1.23}|

\item Pour les courbes, importer des images générées par \texttt{python}

ou bien utiliser le package \texttt{pgfplots} (exemple plus loin).

\item Pour placer les objets où vous le souhaitez, deux solutions sont présentées sur les slides suivants.

\end{itemize}

\end{frame}

%=====================================================
%=====================================================
%=====================================================

%\TPshowboxesfalse % décommenter pour faire disparaitre les boites

\begin{frame}

\frametitle{Première solution (simple) : utiliser textblock}
\framesubtitle{Hello}

\begin{textblock}{115}(0,7)
pour avoir de la doc : \quad \url{https://ctan.org/pkg/textpos}
\end{textblock}

\begin{textblock}{50}(0,30)
pour que les bords des boites ne soient \textbf{pas affichés},
ajouter $\backslash$\texttt{TPshowboxesfalse}
\end{textblock}


\begin{textblock}{50}(55,20)
\begin{center}
Mon image :

\vspace{3mm}

% \includegraphics[width=30mm]{Tux}
\end{center}
\end{textblock}


\end{frame}

%=====================================================
%=====================================================
%=====================================================

\newcommand {\horloge} [4] [black] {% temps X Y
\def \R {.4}
\begin{scope}[xshift=#3cm,yshift=#4cm,color=#1]
\draw[thick,fill=white] (0,0) circle (\R);
\draw[fill]  (0,0) circle (.03);
\foreach \n in {0,30,...,330} {\draw (\n:\R)--(\n:.32);}
\foreach \n in {0,90,...,270} {\draw[thick] (\n:\R)--(\n:.3);}
\draw [thick,->,>=latex] (0,0)--({90-#2*6}:\R);
\end{scope}}

%=====================================================

\begin{frame}

\frametitle{Deuxième solution (très puissante) : utiliser TikZ}
\framesubtitle{Good bye}

\vspace{-1mm}

\hspace*{-8mm}
\begin{tikzpicture}

\node[anchor=text] at (-1.5,3) {pour avoir de la doc : \quad \url{https://ctan.org/pkg/pgf}};

\node[anchor=text] at (-1.5,2.3) {en français : \quad \url{http://math.et.info.free.fr/TikZ}};


% \node at (5,0) {\includegraphics[width=30mm]{Tux}};
\node at (5,-2) {Figure 1};

\uncover<2->{%======================================
\coordinate (A) at (7,-1);
\draw[red,line width = 1.2, ->, >=latex] (A)--(5.3,.1);
\draw (A) ++ (.2,-.2) node[red,anchor=text] {Bec de Tux};
}%===================================================


\uncover<2-2>{%======================================
\node[anchor=text] at (-1,-3.3) {\parbox{110mm}{En physique, \textcolor{red}{il faut mettre des flèches} sur vos photos et vos schémas pour désigner les différents éléments : \textbf{aidez le jury} à bien rentrer dans votre protocole expérimental !}};

}%===================================================


\uncover<3->{%======================================

\begin{scope}[thick]
\def \a {1.5}
\draw[blue,fill=blue!10] (-\a,-\a) rectangle (\a,\a);
\draw[red]  (-\a,0)--(0,\a)--(\a,0)--(0,-\a)--cycle;
\end{scope}

\horloge[violet] {30} {0} {0}
\horloge {5} {-1} {-2.5}
\horloge {0}  {0} {-2.5}
\horloge {-5} {1} {-2.5}
}%===================================================


\end{tikzpicture}

\end{frame}

%=====================================================
%=====================================================
%=====================================================

\begin{frame}

\frametitle{Placez les objets où vous voulez}
\framesubtitle{Il suffit de \frquote{copier-coller} le code de ce slide (et tikzgrille qui est dans le préambule)}

\begin{tikzgrille}[1]
% 1 pour afficher la grille / 0 pour ne pas l'afficher = option par défaut

\def \xt {-4}
\def \yt {2}
\node at (\xt,\yt)   {Une petite grille};
\node at (\xt,\yt-.5) {pour vous aider.};

\node[anchor=text] at (1,-1) {\parbox[t]{60mm}{%
Liste des objectifs :
\begin{itemize}
\item premier objectif
\item deuxième objectif
\item troisième objectif
\end{itemize}
}};

% \node at (-2,-2) {\includegraphics[width=25mm]{Tux}};

\end{tikzgrille}

\end{frame}

%=====================================================
%=====================================================
%=====================================================

\begin{frame}

\frametitle{Pour tracer des courbes}
\framesubtitle{Utiliser le package \texttt{pgfplots}}

\vspace{3mm}

\begin{tikzpicture}
\begin{axis}[% Quelques options parmi un nombre bien plus grand :
%
width = 10cm, height = 6cm,
%
domain=-1:1, samples=60, smooth, % peut aussi se mettre sur chaque courbe
%
title  = {Fonctions d'ondes dans un puits infini},
xlabel = {$x/a$},
ylabel = $\phi(x)$,
%
%axis x line = middle, % = bottom % un seul axe des abscisses au milieu
%axis y line = center, % un seul axe des ordonnées au centre
%
%axis equal,           % Ymax - Ymin = Xmax - Xmin
%enlargelimits = false,  % il n'augmente pas la taille du cadre
%xmin = -1, xmax = 1,  % pas nécessaire avec "axis x line = middle"
%ymin = -1, ymax = 1,
%xtickmin = -.7, xtickmax = .7, % valeurs au dela desquelles il ne met rien (existe aussi sur y)
%minor tick num = 3,     % ajoute des sous graduation (aussi sur grid)
%tick align = outside, % met les tick à l'extérieur
%tickpos = left,       % ne met les tick que en bas et à gauche
grid,            % "= both" si on veut aussi sur minor tick
%
xtick = {-1,-.5,...,1},
xticklabels = {$-1$, $-\frac 12$, $0$, $\frac 12$, $1$},
%
%extra x ticks = {-.3},  % good avec grid
%extra x tick style = {grid style={black}, xticklabel=\empty},
%
legend pos = south west,
]

\addplot [red, line width=1,]
{sin((x+1)*180/2)};

\addplot [blue, line width=1,]
{sin((x+1)*180)};

\legend{$\phi_1(x)$, $\phi_2(x)$}

\end{axis}

\node at (4,-1.8) {\Large Les graduations doivent être \textbf{lisibles}};
\node[red] at (4,-2.4) {\tiny par défaut avec python elles sont \textbf{trop petites} : pensez à les agrandir};

\end{tikzpicture}

\end{frame}

%=====================================================
%=====================================================
%=====================================================

\begin{frame}

\frametitle{Pour vos circuits électroniques}
\framesubtitle{Utilisez par exemple le package maison \texttt{tikzelec}}

\begin{center}
\begin{tikzelec}

\def \XA {0}
\def \XB {2.2}
\def \XC {3.75}
\def \XD {5.3}
\def \XE {7.5}

\def \YA {0}
\def \YB {2}
\def \YC {4}
\def \YM {-.7}

\draw (\XA,\YA) rectangle (\XE,\YC) ;

\TikzelecSourceIdeale    [90] {\XA} {\YB}
\TikzelecTensionComposant[90] {-1}  {0}
\TikzelecNom {$E$}              {-.8} {0}

\TikzelecResistance {\XB} {\YC}
\TikzelecNom {$R$}    {0}   {1}

\TikzelecBobine  {\XD} {\YC}
\TikzelecNom {$L$} {0}   {1}

\TikzelecCondensateur    [90] {\XE} {\YB}
\TikzelecTensionComposant[90] {1} {0}
\TikzelecNom {$u_C$}          {1} {0}
\TikzelecNom {$C$}         {-2.2} {0}

\draw (\XC,\YA)--(\XC,\YM);
\TikzelecMasse {\XC} {\YM}

\node at (\XC,-3.1) {Ne perdez trop pas de temps à essayer de comprendre};
\node at (\XC,-4.3) {comment ça marche : venez me poser des questions !};

\end{tikzelec}
\end{center}

\end{frame}

%=====================================================
%=====================================================
%=====================================================

\end{document}
