\documentclass{article}

\usepackage[french]{babel}
\usepackage[utf8]{inputenc}
\usepackage[T1]{fontenc}

\usepackage[locale = FR]{siunitx} % les unités PROPRES
\sisetup{inter-unit-product = \ensuremath{{}\cdot{}}}

\newcommand{\eff}{_\text{eff}}

\title{Modélisation d'un ordinateur\\
\small{par un cicruit simple}}
\author{Groupe LTB}

\begin{document}

\maketitle

\section{Objectifs}

On cherche à modéliser le comportement d'un ordinateur en consommation de courant par un circuit électrique simple. Les données viennent du TP-1, c'est-à-dire avec $\varphi = \varphi(U\eff) - \varphi(I\eff)$ :

$$U\eff = \SI{230}{V} \quad I\eff = \SI{0,20}{A} \quad \text{et} \quad \cos \varphi = \SI{0,7}{}$$

\section{Modélisation}

On considère 3 modèles :
\begin{enumerate}
    \item Résistance et consensateur en série
    \item Résistance et consensateur en série
    \item Résistance en dérivation sur un consensateur
\end{enumerate}

\section{Résolution}

Les calculs menés pour le modèle 1 donnent $R = \SI{1,6}{k\ohm}$ et $C = \SI{1,9}{\micro F}$, et par ailleurs $\varphi = \SI{-0,8}{rad}$.
Le modèle 2 donne $R = \SI{800}{\ohm}$ et $L = \SI{2,6}{H}$, avec $\varphi = \SI{-0,8}{rad}$.
Le modèle 3 donne quant à lui $R = \SI{1,6}{k\ohm}$ et $C = \SI{2}{\micro F}$, avec $\varphi = \SI{0,8}{rad}$.

\end{document}
