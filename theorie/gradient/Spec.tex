\documentclass{article}

\usepackage[french]{babel} % langue
\usepackage[utf8]{inputenc}
\usepackage[T1]{fontenc}
\usepackage{xspace} % espacement
\usepackage{physics} % dérivées
\usepackage{amssymb}
\usepackage{graphicx}

\usepackage{siunitx}
\sisetup{
    detect-all,
    output-decimal-marker={,},
    group-minimum-digits = 3,
    group-separator={~},
    number-unit-separator={~},
    inter-unit-product={~}
}
\newcommand{\cel}{\degreeCelsius}

\usepackage{pythonhighlight} % coloration
\definecolor{keywordcolour}{HTML}{5e81ac}
\definecolor{literatecolour}{HTML}{5e81ac}
\definecolor{stringcolour}{HTML}{008000}
\usepackage{listingsutf8}
\lstset{
    basicstyle=\small\ttfamily,
    columns=flexible,
    breaklines=true
}

\usepackage{tikz}
\definecolor{nord10}{RGB}{94,129,172}
\definecolor{nord11}{RGB}{191,97,106}

\newcommand{\p}{\texttt} % Stylisé comme du code.
\newcommand{\air}{$\Lambda$\xspace}
\newcommand{\cair}{C_{\Lambda}}
\newcommand{\tair}{T_{air}}
\newcommand{\ordi}{$\Sigma$\xspace}
\newcommand{\etdt}{entre $t$ et $t + \dd t$\xspace}
\newcommand{\exdx}{entre $x$ et $x + \dd x$\xspace}

\title{Algorithme du gradient : cahier des charges}
\author{Malo Leroy -- Ulysse Tanguy--Bompard}

\begin{document}

\maketitle

\section{Objectifs}

On cherche à minimiser la puissance totale consommée par plusieurs ordinateurs dans une même pièce (un \textit{datacenter}) à quantité de calculs constants, en fonction de la répartition spatiale des ordinateurs et de la répartition de la quantité de calculs.

On va utiliser pour cela l'\textbf{algorithme du gradient}, qui sert à minimiser une fonction $f(x_1, ..., x_n)$ en suivant la plus forte pente.

\section{Paramètres}

On considère que nos ordinateurs sont situés dans $n$ armoires, qui comptent chacune $p$ ordinateurs (soit $np$ ordinateurs au total). Chaque armoire est paramétrée par sa position au sol $(X_i, X_i)$ pour $i \in \{1, ..., n\}$, et chaque ordinateur est paramétré par sa quantité de calculs $K_j$ pour $j \in \{1, ..., np\}$.

Pour que l'algorithme du gradient fonctionne, on va introduire des coordonnées réduites, afin de comparer de manière pertinente les paramètres et leurs variations entre aux. Ainsi, à la place des coordonnées spatiales $(X_i, Y_i)$, on utilisera $(x_i, y_i)$, où $x_i = \frac{X_i}{L_x}$ et $y_i = \frac{Y_i}{L_y}$, où $L_x$ et $L_y$ sont les longueurs de la pièce. De même, à la place de la quantité de calcul $K_j$, on utilisera la quantité de calcul réduite $k_j = \frac{K_j}{K}$, où $K$ est la quantité de calcul totale.

On prend aussi en compte le fait que tous les ordinateurs ne sont pas forcément en fonctionnement. Pour cela, en amont du lancement de l'aglorithme, on fixe $b_{1,1}, ..., b_{1, p}, ..., b_{n, p}$ valant \p{True} ou \p{False}, déterminant pour chaque ordinateur s'il est allumé ou pas. La quantité maximale de calcul par ordinateur étant limitée, le nombre minimal d'ordinateur.

\section{Modèles}

\begin{enumerate}
    \item Les parois de la pièce sont à température constante
    \item L'air extérieur est à température constante
    \item Prolongation du modèle précédent par l'implémentation d'un système de refroidissement
\end{enumerate}

\end{document}
